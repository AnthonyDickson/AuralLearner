\documentclass[10pt,a4paper]{article}
\usepackage[utf8]{inputenc}
\usepackage{amsmath}
\usepackage{amsfonts}
\usepackage{amssymb}
\usepackage{url}
% !TeX spellcheck = en_GB 

\author{Anthony Dickson, Rory Jackson, Johnny Mann}
\title{COSC345 Assignment 2, Alpha Version - Report}

\begin{document}	
\maketitle

\section{Overview}
The goal of our year-long project is to create an Android application that teaches the aural components of music theory, and supports visual impairments that makes one unable to read text. We have managed to implement the majority of functionality that we outlined in the initial report. In this report we will go over what we have implemented in this alpha version, and what difficulties we encountered in the course of developing this alpha version.

\section{Group Members}
\begin{itemize}
	\item Anthony Dickson - 3348967
	\item Rory Jackson - 2208377
	\item Johnny Mann - 3891999
\end{itemize}

\section{Alpha Version Features}
In the time between Assignment 1 and now, we have implemented features which enable the user to:
\begin{itemize}
	\item navigate the app with a basic GUI.
	\item navigate the app menus with basic voice commands.
	\item have things read out to them via text-to-speech.
	\item run a basic exercise where they can try to sing and match their pitch with a single musical pitch (pitch matching).
\end{itemize}
The features we have implemented so far are quite basic but they serve as a proof of concept and a starting point for us to flesh out features. 

One thing that we were unable to fully implement was the ability to navigate the entire app with voice commands. This was due to the fact that the microphone can only be used by one process/thread at a time, meaning that voice recognition cannot be used in exercises that need to record the user's singing and is only available in the menus.

\section{A Note On Running the App}
When you are opening the project with android studio, please make sure you open the subdirectory 'android/' and not the git repository root directory. Also, our app heavily relies on audio input to function so if you are running our app in an emulator, please make sure the computer you are using has a working microphone.

\section{Difficulties We Encountered}
One of the main difficulties we encountered was programming and coding in an environment that we were not familiar with, in particular creating mobile applications, dealing with voice input and programming multi-threaded applications. This required a substantial amount of time to study these new concepts and acquire the necessary skills to implement the functionality needed for our application, and often required us to rethink our approach to solving issues (e.g. a lot of things in an app have to done and dealt with asynchronously).

Some of the concepts we needed to implement for our app turned out to be more complex than we had thought. For example, the Fast Fourier Transform algorithm is rather complicated. Rather than spend lots of time trying to understand and implement it ourselves, we found that it was far more productive to reuse open source code found on GitHub. This saved us a great deal of time, and taught us the importance of using our time productively and the time savings that come with code reuse.

Communication and conflicting ideas were other difficulties that we had to overcome during this semester. At the start, we had trouble communicating our ideas precisely and accurately, which led to misunderstandings about what we were going to make, what features were needed, and how we wanted to implement certain features. Sometimes we would also have conflicting ideas on how something should be done, which caused delays. We were able to overcome these issues by setting aside time to properly discuss any things that we did not understand, and resolve any conflicting ideas through discussion and forming a general consensus. We believe that through our experience this semester we have learned to better communicate our ideas and work together in a group more efficiently.

Finally, time constraints and time management were other difficulties that we had to deal with. We found that trying to find a good time for everyone to meet up was not always easy since everyone had different schedules. One member was busy since they were taking five papers this semester and all of us spent quite a bit time on our other commitments. These factors meant that we were unable to strictly follow the schedule we had planned out, and work on this assignment became rather sporadic. Although this is an issue that we are still working on, it has taught us the importance of trying to manage our time more efficiently and to use our time more wisely.
\end{document}